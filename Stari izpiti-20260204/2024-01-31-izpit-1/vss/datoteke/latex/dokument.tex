\documentclass[a4paper]{article}
\usepackage[slovene]{babel}
\usepackage{amsfonts,amssymb,amsmath,mathrsfs,amsthm}
\usepackage[utf8]{inputenc}
\usepackage[T1]{fontenc}
\usepackage{url}
\usepackage{hyperref}

%%%%%%%%%%%%%%%%%%%%%%%%%%%%%%%%%%%%%%%%%%%%%%%%%%%%%%%%%%%%%%%%%%%%%%%%%%%%%

\def\R{\mathbb{R}}  % mnozica realnih stevil

%%%%%%%%%%%%%%%%%%%%%%%%%%%%%%%%%%%%%%%%%%%%%%%%%%%%%%%%%%%%%%%%%%%%%%%%%%%%%

% Thomaeova funkcija
% Beno Učakar

\begin{document}

Oglejmo si Thomaeovo funkcijo, ki je primer funkcije prvega Bairovega razreda.
Take funkcije lahko definiramo na naslednji način.
Naj bo ??. Funkcija ?? je \emph{funkcija prvega Bairovega razreda}, 
če obstaja funkcijsko zaporedje $\{f_n\}$ zveznih funkcij na $D$, ki po točkah konvergira k $f$. 
Ta razred označimo z $b1(D)$ oziroma, če ne bo nevarnosti zmede, kar z $b1$.

% začetek zgleda
    Funkcijsko zaporedje ?? definiramo na sledeč način.
    Za vsak ??, $1 \le q < n$ in $0 \le p \le q$ definiramo
    \begin{itemize}
        \item \(f_n(x) = \max\left\{\frac{1}{n}, \frac{1}{q} + 2n^2\left(x - \frac{p}{q}\right)\right\}\) na intervalu \(\left(\frac{p}{q} - \frac{1}{2n^2}, \frac{p}{q}\right)\) in
        \item \(f_n(x) = \max\left\{\frac{1}{n}, \frac{1}{q} - 2n^2\left(x - \frac{p}{q}\right)\right\}\) na intervalu \(\left(\frac{p}{q}, \frac{p}{q} + \frac{1}{2n^2}\right)\).
    \end{itemize}
    V vseh ostalih točkah naj bo $f_n(x) = \frac{1}{n}$.
    Preverimo lahko, da so intervali $\left(\frac{p}{q} - \frac{1}{2n^2}, \frac{p}{q} + \frac{1}{2n^2}\right)$ paroma disjunktni in zgornja definicija je dobra.
    Opazimo, da je $f_n(x)$ odsekoma linearna zvezna. Če vzamemo limito po točkah, dobimo

    ??

    Pokazali smo, da \emph{Thomaeova funkcija} pripada $b1$. 
% konec zgleda

%%%%%%%%%%%%%%%%%%%%%%%%%%%%%%%%%%%%%%%%%%%%%%%%%%%%%%%%%%%%%%%%%%%%%

\end{document}