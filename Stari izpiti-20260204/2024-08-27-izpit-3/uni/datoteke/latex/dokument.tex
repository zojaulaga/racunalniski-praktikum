\documentclass[a4paper, 11pt]{article}
\usepackage[utf8]{inputenc}
\usepackage[T1]{fontenc}
\usepackage[slovene]{babel}
\usepackage{lmodern}

\usepackage{amsmath}
\usepackage{amsthm}
\usepackage{amsfonts}

\newcommand{\N}{\mathbb{N}}

\theoremstyle{definition}
\newtheorem{definicija}{Definicija}

% Peanovi aksiomi
% Mojca Novak

\begin{document}

\title{Peanovi aksiomi}
\author{Mojca Novak}
\date{}
\maketitle

\begin{abstract}
        V nadaljevanju je zapisana definicija Peanovih aksiomov.
\end{abstract}

     \begin{definicija}
        \emph{Peanovi aksiomi}
        Množica naravnih števil je množica $\N$ s funkcijo ??, 
        ki vsakemu naravnemu številu $\N$ priredi njegovega neposrednega naslednika $\varphi(n)$. 
        Pri tem veljajo naslednji aksiomi:
        \begin{enumerate}
            \item $\N$ vsebuje število $\epsilon$, ki ni neposredni naslednik nobenega naravnega števila;
            \item neposredna naslednika dveh različnih naravnih števil sta različna, tj.\ funkcija $\varphi$ je injektivna: 
            če je ??, je ??;
            \item Če za podmnožico ?? veljata lastnosti:
            \begin{enumerate}
                \item ?? in
                \item če je ??, je tudi ??,
                potem je $A = \N$.
            \end{enumerate}
        \end{enumerate}
    \end{definicija} 
        % konec definicije

Povzeto po zapiskih \cite{slika} in \cite{zapiski}.
\bibliographystyle{plain}
\bibliography{viri}   
    
\end{document}